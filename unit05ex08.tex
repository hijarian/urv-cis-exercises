\section{Unit 5 Exercise 8}

\subsection{Task}
AES in practice.

\begin{enumerate}
    \item[a)] Search for applications using AES, and explain how they use it.
    \item[b)] In some of these applications, AES is used in modes that are different from the ones we saw in the course (we saw ECB, CBC, CFB, CTR). Choose one of these modes, explain how it works, and explain its advantages.
\end{enumerate}

At least 1 page

\subsection{Solution}

An AES ``mode'' is a way to use AES, which is a block cipher, to encrypt data that is longer than a single block.
Different modes have different trade-offs, thus, different applications use different modes for their specific use case.

As an example, WhatsApp uses \textit{three} different modes for different purposes, as explained in their security whitepaper~\cite{whatsapp}:
\begin{itemize}
    \item For the "link using an 8-character code" feature, they use AES-GCM and AES-CTR.
    \item ``Once a session has been established, clients exchange messages that are protected with a Message Key using AES256 in CBC mode for encryption [and HMAC-SHA2256 for authentication]'', end quote from the whitepaper, page 15.
\end{itemize}

A mode which is the industry standard for disk data encryption is XTS\@.
This was formally declared in 2010 by the NIST Special Publication 800-38E~\cite{NIST-XTS}.

While the NIST 800-38E is a great explanation of the mode by itself, let's comment on it shortly here.

XTS mode is based on the paper by Rogaway, which introduces the concept of \textit{tweakable block ciphers}~\cite{Rogaway}.

VeraCrypt~\cite{VeraCrypt} is an open-source disk encryption software which uses XTS mode.
In its documentation there's a succinct description of the mode~\cite{VeraCrypt-modes}:

\begin{fancyquote}
    Description of the XTS mode:
    \begin{align*}
        C_i &= E_{K1}(P_i \wedge (E_{K2}(n) \otimes a^i)) \wedge (E_{K2}(n) \otimes a^i)
    \end{align*}
    Where:
    \begin{itemize}
        \item $\otimes$ denotes multiplication of two polynomials over the binary field $GF(2)$ modulo $x^{128} + x^7 + x^2 + x + 1$.
        \item $K1$ is the encryption key (256-bit for each supported cipher; i.e, AES, Serpent, and Twofish)
        \item $K2$ is the secondary key (256-bit for each supported cipher; i.e, AES, Serpent, and Twofish)
        \item $i$ is the cipher block index within a data unit; for the first cipher block within a data unit, $i = 0$.
        \item $n$ is the data unit index within the scope of $K1$; for the first data unit, $n = 0$.
        \item $a$ is a primitive element of Galois Field $GF(2^{128})$ that corresponds to polynomial $x$ (i.e., 2).
    \end{itemize}
    The size of each data unit is always 512 bytes (regardless of the sector size).
\end{fancyquote}

In the quote above:
\begin{itemize}
    \item $C_i$ is the $i$-th cipher block.
    \item $P_i$ is the $i$-th plaintext block.
    \item $E_{K1}$ and $E_{K2}$ denote the encryption function with keys $K1$ and $K2$, respectively.
    \item $\wedge$ denotes the XOR operation.
\end{itemize}

The core point of the whole process is that the number of the block is involved in the encryption process.
This idea allows for a very efficient parallelization of the encryption and decryption processes as we essentially enable random access to the data.

The implementation of XTS can be found in the Linux kernel~\cite{linux-kernel}, in the file \texttt{crypto/xts.c} (https://github.com/torvalds/linux/blob/master/crypto/xts.c).

While Rogaway's paper~\cite{Rogaway} proves that the mode preserves the security properties of the underlying cipher, it still have inherent downsides.

\begin{itemize}
    \item XEX and by extension XTS does not provide \textit{authenticity}. If the attacker modifies a block in the ciphertext, the decryption will produce a valid plaintext with mangled data. In addition to that, it enables \textit{watermarking attacks}.
    \item This is a fixed-size block cipher mode, and as such, it lacks the flexibility of other modes, such as GCM, which is able to operate on arbitrary-length messages naturally.
\end{itemize}

In addition to that, the NIST-approved XTS based on IEEE P1619~\cite{IEEE1619} specifically have the following peculiarities:
\begin{itemize}
    \item Original XEX works with the fixed-length data blocks \textit{only}.
    To circumvent it, XTS introduces the technique of \textit{ciphertext stealing}, which essentially allows for the last block to be of a different size.
    Implementation of this technique is non-trivial and increases the cost and complexity of the mode implementation.
    \item The AES key length is limited to only two options: 128-bit (for 256-bit AES) or 256-bit (for 512-bit AES).
\end{itemize}

In layman terms, XTS is a very good mode for disk encryption, because it's better than raw CBC, but it's not the best mode for all use cases, as GCM is significantly more feature-rich.

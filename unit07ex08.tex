\chapter*{Unit 7}
\section*{Exercise 8}

\subsection*{Task}

Diffie Hellman KE protocol.

Compute an instance of the protocol with $G = (\mathbb{Z}_{13}^*, \cdot)$


\subsection*{Solution}

To start the protocol, Alice and Bob agree on a generator $g$ and a modulus $q$.

Since 13 is a prime number, the set $\mathbb{Z}_{13}^*$ contains all numbers from 1 to 12.

\begin{equation}
    \mathbb{Z}_{13}^* = \{1, 2, 3, 4, 5, 6, 7, 8, 9, 10, 11, 12\}
\end{equation}

Therefore, 

\begin{equation}
    q = 12
\end{equation}

To find a generator, we can choose any number from 1 to 12 such that its powers generate all numbers from 1 to 12 without repetition.

Let's try $g = 2$:

\begin{align}
1 &= 2^{12} &= 4096 &= 1 \mod 13 \\
2 &= 2^1    &= 2    &= 2 \mod 13 \\
3 &= 2^4    &= 16   &= 3 \mod 13 \\
4 &= 2^2    &= 4    &= 4 \mod 13 \\
5 &= 2^9    &= 512  &= 5 \mod 13 \\
6 &= 2^5    &= 32   &= 6 \mod 13 \\
7 &= 2^{11} &= 2048 &= 7 \mod 13 \\
8 &= 2^3    &= 8    &= 8 \mod 13 \\
9 &= 2^8    &= 256  &= 9 \mod 13 \\
10 &= 2^{10} &= 1024 &= 10 \mod 13 \\
11 &= 2^7    &= 128  &= 11 \mod 13 \\
12 &= 2^6    &= 64   &= 12 \mod 13
\end{align}

Thus, $2$ is a generator for $\mathbb{Z}_{13}^*$, so:

\begin{equation}
    g = 2
\end{equation}

Alice has collected the three required items to start the handshake according to the protocol:

\begin{equation}
    (G = (\mathbb{Z}_{13}^*, \cdot), q = 12, g = 2)
\end{equation}

Alice chooses a secret key $a \in \mathbb{Z}_{13}^*$:

\begin{equation}
    a = 5
\end{equation}

Alice computes her side of challenge (a public key): 

\begin{equation}
    A = g^a \mod 13 = 2^5 \mod 13 = 32 \mod 13 = 6
\end{equation}

Alice sends her part of handshake to Bob: 

\begin{equation}
    (G = (\mathbb{Z}_{13}^*, \cdot), q = 12, g = 2, A = 6)
\end{equation}

Bob chooses a secret key $b \in \mathbb{Z}_{13}^*$:

\begin{equation}
    b = 7
\end{equation}

Bob computes his side of challenge (a public key):

\begin{equation}
    B = g^b \mod 13 = 2^7 \mod 13 = 128 \mod 13 = 11
\end{equation}

Bob sends his part of handshake to Alice:

\begin{equation}
    (G = (\mathbb{Z}_{13}^*, \cdot), q = 12, g = 2, B = 11)
\end{equation}

Alice computes shared secret key $s$:

\begin{equation}
    s = B^a \mod 13 = 11^5 \mod 13 = 161051 \mod 13 = 9
\end{equation}

Bob computes shared secret key $s$:

\begin{equation}
    s = A^b \mod 13 = 6^7 \mod 13 = 279936 \mod 13 = 9
\end{equation}

Thus, now Alice and Bob share the secret key $s = 9$.

\section{Summary}

For those who have full understanding of the context of the paper, the short summary is this.

This paper rigorously defines as proven theorems the bounds for the share sizes for the following types of secret-sharing schemes:

\begin{itemize}
    \item Secret-sharing schemes of almost all access structures
    \item Graph secret-sharing schemes
\end{itemize}

Also, for all very dense graphs, it improves the total share size.

Everything further below is a more detailed review with explanations of the context and the discussion of this results.

It is assumed that this review will be read with the original paper alongside.
When necessary, the relevant parts of the original paper will be referred to, not cited.

Through the whole paper a number $n$ is used.
It's crucial to remember that $n$ is not the length of the secret itself but the number of parties (or amount of shares) to which the secret is split.

\section{Asymptotic Notation}

The paper uses extensively the asymptotic notation to describe the bounds of the share sizes.
It is absolutely necessary to understand the notation to understand the paper.

The good reference to almost all of it is the book by Cormen et al. \cite{cormen2009introduction}.

When the asymptotic notation is used as part of a mathematical expression, it means an arbitrary function that would satisfy the condition.

Let's summarize the notation used in the paper.

\subsection{Bounded}

\[
    \Theta\left(g\left(n\right)\right)= \{f(n) | \exists c_1 \exists c_2 \exists n_0: \forall n \ge n_o: 0 \le c_1 g(n) \le f(n) \le c_2 g(n) \}
\]

That is, $\Theta$ is the set of functions $f(n)$ that are bounded both from above and below by $g(n)$ scaled by two constants $c_1, c_2$.

\subsection{Upper bound}

\[
    O\left(g\left(n\right)\right)={f(n): \exists c \exists n_0: \forall n \ge n_o: 0 \le  f(n) \le c g(n)}
\]

That is, $O$ is the set of functions $f(n)$ that are bounded from above by $g(n)$ with respect to a constant factor $c$.

\subsection{Lower bound}

\[
    \Omega\left(g\left(n\right)\right)={f(n): \exists c \exists n_0: \forall n \ge n_o: 0  \le c g(n) \le  f(n)}
\]

That is, $\Omega$ is the set of functions $f(n)$ that are bounded from below by $g(n)$ with respect to a constant factor $c$.

\subsection{Upper bound asymptotically not tight}

\[
    o(g(n)) = f(n): \forall c > 0: \exists n_o: 0 \le f(n) \le cg(n), \forall n \ge n_o
\]

That is, $\lim_{n\rightarrow\infty}\frac{f(n)}{g(n)}=0$.
As we move further along $n$, $f(n)$ becomes insignificant compared to $g(n)$. 

\subsection{Lower bound asymptotically not tight}

\[
    \omega(g(n)) = f(n): \forall c > 0: \exists n_o: 0  \le cg(n)\le f(n), \forall n \ge n_o
\]

That is, $\lim_{n\rightarrow\infty}\frac{f(n)}{g(n)}=\infty$.
As we move further along $n$, $g(n)$ becomes insignificant compared to $f(n)$.

\subsection{Soft-O}

An additional notation which is used in this paper extensively but is not covered in \cite{cormen2009introduction} is the $\tilde{O}$ notation.

The $\tilde{O}$ notation is used to hide polylogarithmic factors, that is:

\[
    f(n) \in \tilde{O}(g(n)) \Longleftrightarrow f(n) \in O(g(n) \cdot \text{polylog}(n))
\]

where $\text{polylog}(n) = \log^c n$ for some constant $c$.

\section{Secret Sharing Schemes}

A secret-sharing scheme can be represented symbolically as follows:

\[
    S \times R \rightarrow S_1 \times S_2 \times \cdots \times S_n
\]

Where $S$ is the non-empty non-singular set of secrets, $R$ is the set of random strings, and $S_1, S_2, \ldots, S_n$ are the set of shares of the secret for a set of parties $P$.

A party $P_i$ has an ability to own shares from $S_i$.

A secret-sharing scheme assume existence of a trusted \textit{dealer} who distributes shares to the parties.

An access structure is a collection of sets of parties that can reconstruct the secret.
It is represented in the paper by $\Gamma$.


arbitrary access structure
graphs
forbidden graphs

vs 

"a" share scheme
linear scheme
multi-linear scheme

Two families of secret-sharing schemes which are the core topic in the paper:
1. graph secret-sharing schemes 
2. multi-linear secret-sharing schemes

\subsection{Sizes of shares}

The paper stresses out the distinction between the total size of the shares and maximum sizes of the shares.


\section{Graph Sharing Schemes}


\section{CDS}
\section{Main results}

The paper rigorously defines the bounds for the share sizes for the following types of secret-sharing schemes:

\section{Discussion of the results}

authors perform super rigorous analysis on an array of types of SSS

the paper is super tight on math so no detailed analysis it will take too much time

this paper can also serve as an overview on the major SSS types

core topic are graph secret-sharing schemes and CDS

for some cases they were able to determine the bounds more precisely

\section{Summary}

For those who have full understanding of the context of the paper, the short summary is this.

This paper rigorously defines as proven theorems the bounds for the share sizes for the following types of secret-sharing schemes:

\begin{itemize}
    \item Secret-sharing schemes of almost all access structures
    \item Graph secret-sharing schemes
\end{itemize}

Also, for all very dense graphs, it improves the total share size.

Everything further below is a more detailed review with explanations of the context and the discussion of this results.

It is assumed that this review will be read with the original paper alongside.
When necessary, the relevant parts of the original paper will be referred to, not cited.

Through the whole paper a number $n$ is used.
It's crucial to remember that $n$ is not the length of the secret itself but the number of parties (or amount of shares) to which the secret is split.

\section{Asymptotic Notation}

The paper uses extensively the asymptotic notation to describe the bounds of the share sizes.
It is absolutely necessary to understand the notation to understand the paper.

The good reference to almost all of it is the book by Cormen et al. \cite{cormen2009introduction}.

When the asymptotic notation is used as part of a mathematical expression, it means an arbitrary function that would satisfy the condition.

Let's summarize the notation used in the paper.

\subsection{Bounded}

\[
    \Theta\left(g\left(n\right)\right)= \{f(n) | \exists c_1 \exists c_2 \exists n_0: \forall n \ge n_o: 0 \le c_1 g(n) \le f(n) \le c_2 g(n) \}
\]

That is, $\Theta$ is the set of functions $f(n)$ that are bounded both from above and below by $g(n)$ scaled by two constants $c_1, c_2$.

\subsection{Upper bound}

\[
    O\left(g\left(n\right)\right)={f(n): \exists c \exists n_0: \forall n \ge n_o: 0 \le  f(n) \le c g(n)}
\]

That is, $O$ is the set of functions $f(n)$ that are bounded from above by $g(n)$ with respect to a constant factor $c$.

\subsection{Lower bound}

\[
    \Omega\left(g\left(n\right)\right)={f(n): \exists c \exists n_0: \forall n \ge n_o: 0  \le c g(n) \le  f(n)}
\]

That is, $\Omega$ is the set of functions $f(n)$ that are bounded from below by $g(n)$ with respect to a constant factor $c$.

\subsection{Upper bound asymptotically not tight}

\[
    o(g(n)) = f(n): \forall c > 0: \exists n_o: 0 \le f(n) \le cg(n), \forall n \ge n_o
\]

That is, $\lim_{n\rightarrow\infty}\frac{f(n)}{g(n)}=0$.
As we move further along $n$, $f(n)$ becomes insignificant compared to $g(n)$. 

\subsection{Lower bound asymptotically not tight}

\[
    \omega(g(n)) = f(n): \forall c > 0: \exists n_o: 0  \le cg(n)\le f(n), \forall n \ge n_o
\]

That is, $\lim_{n\rightarrow\infty}\frac{f(n)}{g(n)}=\infty$.
As we move further along $n$, $g(n)$ becomes insignificant compared to $f(n)$.

\subsection{Soft-O}

An additional notation which is used in this paper extensively but is not covered in \cite{cormen2009introduction} is the $\tilde{O}$ notation.

The $\tilde{O}$ notation is used to hide polylogarithmic factors, that is:

\[
    f(n) \in \tilde{O}(g(n)) \Longleftrightarrow f(n) \in O(g(n) \cdot \text{polylog}(n))
\]

where $\text{polylog}(n) = \log^c n$ for some constant $c$.

\section{Secret Sharing Schemes}

A secret-sharing scheme can be represented symbolically as follows:

\[
    S \times R \rightarrow S_1 \times S_2 \times \cdots \times S_n
\]

Where $S$ is the non-empty non-singular set of secrets, $R$ is the set of random strings, and $S_1, S_2, \ldots, S_n$ are the set of shares of the secret for a set of parties $P$.

A party $P_i$ has an ability to own shares from $S_i$.

A secret-sharing scheme assume existence of a trusted \textit{dealer} who distributes shares to the parties.

\subsection{Access Structures}

An access structure is a collection of sets of parties that can reconstruct the secret.
It is represented in the paper by $\Gamma$.


arbitrary access structure
graphs
forbidden graphs

vs 

"a" share scheme
linear scheme
multi-linear scheme

Two families of secret-sharing schemes which are the core topic in the paper:
1. graph secret-sharing schemes 
2. multi-linear secret-sharing schemes

\subsection{Sizes of shares}

The paper stresses out the distinction between the total size of the shares and maximum sizes of the shares.

The \textbf{total size} of the shares is the sum of the sizes of all shares held by all parties in a given secret-sharing scheme.

The \textbf{maximum size} of the shares is the maximum size of the shares over all parties in a given secret-sharing scheme.

Formally the definition given in the paper is:

\emph{Given a secret-sharing scheme $\Pi$,
define the \emph{size} of the secret as $\log |S|$,
the \emph{share size} of party $P_j$ as $\log |S_j|$,
the \emph{maximum share size} as $\max_{1\le j \le n} \{\log |S_j| \}$,
and the \emph{total share size} as $\sum_{j=1}^{n} \log |S_j|$.
}

\section{Graph Sharing Schemes}

\section{CDS}
\section{Main results}

The paper rigorously defines the bounds for the share sizes for the number of secret-sharing schemes.

For the secret-sharing schemes of \textbf{almost all access structures}, the paper proves the following bounds:

\begin{itemize}
    \item The maximum share size for one-bit secret is bounded 
    \begin{itemize}
        \item from the top by $2^{O(\sqrt{n}\log n)}$
        \item from the bottom by $\Omega(\log n)$
    \end{itemize}
    \item The maximum share size for linear schemes is bounded
    \begin{itemize}
        \item from the top by $2^{n/2+o(n)}$
        \item from the bottom by $\Omega(2^{n/3-o(n)}\log q)$
    \end{itemize}
    \item The normalized maximum share size for multi-linear schemes for secrets of size $2^{n^2}$ is bounded
    \begin{itemize}
        \item from the top by $2^{O(\sqrt{n \log n})}$
    \end{itemize}
\end{itemize}

All these results are specified as mathematically proven theorems.

While proving these bounds, two hypotheses from \cite{applebaumAmortization} are proven:

\begin{itemize}
    \item \textbf{SS is short}. Every access structure over n parties is realizable with small information ratio (say $2^{o(n)}$ ).
    \item \textbf{SS is amortizable}. For every access structure over $n$ parties, and every sufficiently long secret $s$, 
            there exists a secret-sharing scheme with small information ratio (e.g., sub-exponential in $n$).
\end{itemize}

For the secret-sharing schemes of \textbf{almost all graphs}, the paper proves the following bounds:

\begin{itemize}
    \item The maximum share size for one-bit secret is bounded 
    \begin{itemize}
        \item from the top by $n^{o(1)}$
        \item from the bottom by $\Omega(\log n)$
    \end{itemize}
    \item The maximum share size for linear schemes is bounded
    \begin{itemize}
        \item from the top by $\tilde{O}(\sqrt{n} \log q)$
    \end{itemize}
    \item The normalized maximum share size for multi-linear schemes for secrets of size $2^{n^2}$ is bounded
    \begin{itemize}
        \item from the top by $\tilde{O}(\log n)$
        \item from the bottom by $\Omega(\log^{1/2} n)$
    \end{itemize}
\end{itemize}

All these results are specified as mathematically proven theorems.

\section{Discussion of the results}

Authors perform super rigorous analysis on an array of types of secret-sharing schemes.
The paper is super tight on mathematics and is obviously targeted on the specialists in cryptography.

However, this paper can also serve as an overview on the major secret-sharing schemes and their properties.

It seriously looks like multiple papers compressed into one, especially with the result for the dense graphs which looks like a separate work on its own.
As the main value of this paper is summarizing the existing works on share sizes and proving some of the results,
but the result for the dense graphs has been gotten by the novel usage of a CDS scheme.

I admit I didn't understand the importance of the term "almost all access structures" used extensively in the paper.

In general, knowing the theoretical bounds for the share sizes is crucial for the practical implementation of the secret-sharing schemes.
As it's important to know your absolute limits to understand when you need to stop optimizing your code.


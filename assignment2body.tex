\section{Summary}

For those who have full understanding of the context of the paper, the short summary is this.

This paper rigorously defines as proven theorems the bounds for the share sizes for the following types of secret-sharing schemes:

\begin{itemize}
    \item Secret-sharing schemes of almost all access structures
    \item Graph secret-sharing schemes
\end{itemize}

Also, for all very dense graphs, it improves the total share size.

Everything further below is a more detailed review with explanations of the context and the discussion of this results.

It is assumed that this review will be read with the original paper alongside.
When necessary, the relevant parts of the original paper will be referred to, not cited.

Through the whole paper a number $n$ is used.
It's crucial to remember that $n$ is not the length of the secret itself but the number of parties (or amount of shares) to which the secret is split.

\section{Asymptotic Notation}

The paper uses extensively the asymptotic notation to describe the bounds of the share sizes.
It is absolutely necessary to understand the notation to understand the paper.

The good reference to almost all of it is the book by Cormen et al. \cite{cormen2009introduction}.

When the asymptotic notation is used as part of a mathematical expression, it means an arbitrary function that would satisfy the condition.

Let's summarize the notation used in the paper.

\subsection{Bounded}

\[
    \Theta\left(g\left(n\right)\right)= \{f(n) | \exists c_1 \exists c_2 \exists n_0: \forall n \ge n_o: 0 \le c_1 g(n) \le f(n) \le c_2 g(n) \}
\]

That is, $\Theta$ is the set of functions $f(n)$ that are bounded both from above and below by $g(n)$ scaled by two constants $c_1, c_2$.

\subsection{Upper bound}

\[
    O\left(g\left(n\right)\right)={f(n): \exists c \exists n_0: \forall n \ge n_o: 0 \le  f(n) \le c g(n)}
\]

That is, $O$ is the set of functions $f(n)$ that are bounded from above by $g(n)$ with respect to a constant factor $c$.

\subsection{Lower bound}

\[
    \Omega\left(g\left(n\right)\right)={f(n): \exists c \exists n_0: \forall n \ge n_o: 0  \le c g(n) \le  f(n)}
\]

That is, $\Omega$ is the set of functions $f(n)$ that are bounded from below by $g(n)$ with respect to a constant factor $c$.

\subsection{Upper bound asymptotically not tight}

\[
    o(g(n)) = f(n): \forall c > 0: \exists n_o: 0 \le f(n) \le cg(n), \forall n \ge n_o
\]

That is, $\lim_{n\rightarrow\infty}\frac{f(n)}{g(n)}=0$.
As we move further along $n$, $f(n)$ becomes insignificant compared to $g(n)$. 

\subsection{Lower bound asymptotically not tight}

\[
    \omega(g(n)) = f(n): \forall c > 0: \exists n_o: 0  \le cg(n)\le f(n), \forall n \ge n_o
\]

That is, $\lim_{n\rightarrow\infty}\frac{f(n)}{g(n)}=\infty$.
As we move further along $n$, $g(n)$ becomes insignificant compared to $f(n)$.

\subsection{Soft-O}

An additional notation which is used in this paper extensively but is not covered in \cite{cormen2009introduction} is the $\tilde{O}$ notation.

The $\tilde{O}$ notation is used to hide polylogarithmic factors, that is:

\[
    f(n) \in \tilde{O}(g(n)) \Longleftrightarrow f(n) \in O(g(n) \cdot \text{polylog}(n))
\]

where $\text{polylog}(n) = \log^c n$ for some constant $c$.

\section{Secret Sharing Schemes}

A secret-sharing scheme $\Pi$ can be represented symbolically as follows:

\[
    \Pi: S \times R \rightarrow S_1 \times S_2 \times \cdots \times S_n
\]

Where $S$ is the non-empty non-singular set of secrets, $R$ is the set of random strings, and $S_1, S_2, \ldots, S_n$ are the set of shares of the secret for a set of parties $P$.

A party $P_i$ has an ability to own shares from $S_i$.

A secret-sharing scheme assume existence of a trusted \textit{dealer} who distributes shares to the parties.

Formally, a dealer distributes a secret $s \in S$
according to $\Pi$ by first sampling a random string $r \in R$ with uniform distribution,
computing a vector of shares $\Pi(s, r) = (s1, \cdots , sn)$, and privately communicating
each share $s_j$ to party $P_j$.

\subsection{Access Structures}

An access structure is a collection of sets of parties that can reconstruct the secret.

That is, if $P = \{P_1, \cdots, P_n\}$ is a set of parties, then the monotone collection $\Gamma \subseteq 2^P$ of non-empty subsets of $P$ is an access structure $\Gamma$.

Sets in $\Gamma$ are called \textbf{authorized} sets, and the sets not in $\Gamma$ are called \textbf{unauthorized} sets.

It is said that a secret-sharing scheme $\Pi$ with domain of secrets $S$ realizes an access structure $\Gamma$ if the following two requirements hold:

\textbf{Correctness}. The secret $s$ can be reconstructed by any authorized set of parties.
That is, for any set $B = \{P_{i_1}, \cdots, P_{i_{|B|}}\} \in \Gamma$ there exists a reconstruction
function $\text{Recon}_B : S_{i_1} \times \cdots \times S_{i_{|B|}} \rightarrow S$ such that

$$\text{ReconB} (\Pi_B (s, r)) = s$$

for every secret $s \in S$ and every random string $r \in R$.

\textbf{Privacy}. Any forbidden set cannot learn anything about the secret from its shares.

Formally, for any set $T = \{P_{i_1}, \cdots, P_{i_{|T|}}\} \notin \Gamma$ and 
every pair of secrets $s, s' \in S$,
the distributions $\Pi_T (s, r)$ and $\Pi_T (s', r)$ are identical,
where the distributions are over the choice of $r$ from $R$ at random with uniform distribution.

\subsection{Linear and multi-linear secret-sharing schemes}

The bounds for the share sizes are very different for linear and multi-linear secret-sharing schemes.

We say that $\Pi$ is a multi-linear secret-sharing scheme over a finite field $\mathbb{F}$ if there are integers
$\mathcal{l}_d, \mathcal{l}_r , \mathcal{l}_1 , \cdots , \mathcal{l}_n$ such that $S = \mathbb{F}^{\mathcal{l}_d} , R = \mathbb{F}^{\mathcal{l}_r} , S_1 = \mathbb{F}^{\mathcal{l}_1} , \cdots , S_n = \mathbb{F}^{\mathcal{l}_n}$,
and the mapping $\Pi$ is a linear mapping over $\mathbb{F}$ from $\mathbb{F}^{\mathcal{l}_d + \mathcal{l}_r}$ to $\mathbb{F}^{\mathcal{l}_1 + \cdots + \mathcal{l}_n}$.
We say that a scheme is linear over $\mathbb{F}$ if $S = \mathbb{F}$ (\textit{i. e.}, when $\mathcal{l}_d = 1$).

\subsection{Sizes of shares}

The paper stresses out the distinction between the total size of the shares and maximum sizes of the shares.

The \textbf{total size} of the shares is the sum of the sizes of all shares held by all parties in a given secret-sharing scheme.

The \textbf{maximum size} of the shares is the maximum size of the shares over all parties in a given secret-sharing scheme.

Formally the definition given in the paper is:

\emph{Given a secret-sharing scheme $\Pi$,
define the \emph{size} of the secret as $\log |S|$,
the \emph{share size} of party $P_j$ as $\log |S_j|$,
the \emph{maximum share size} as $\max_{1\le j \le n} \{\log |S_j| \}$,
and the \emph{total share size} as $\sum_{j=1}^{n} \log |S_j|$.
}

\section{Graph secret-sharing Schemes}

This is one of the subclasses of the secret-sharing schemes which share size bounds are explored in this paper very extensively.

If we have a graph $G$, we say that a secret-sharing scheme realizes this graph $G$ if it realizes the access structure $\Gamma_G$ with the following properties:

\begin{itemize}
    \item minimal authorized subsets are the edges in $G$.
    \item unauthorized sets are independent sets in $G$.
\end{itemize}

\subsection{Forbidden graph access structures}

The great importance in the paper is given to the \textit{forbidden graph access structures} which are the specialization of the graph access structures.

As described in \cite{forbiddenGraphAccessStructures}, the forbidden graph access structure is a graph access structure described as follows:

\begin{itemize}
    \item authorized sets are all edges in the graph
    \item the unauthorized sets are all independent sets in the graph.
    \item \textit{in addition to the above}, the authorized sets are all subsets of vertices of size greater than two
\end{itemize}

\subsection{Very dense graphs}

In this paper a specialization of graphs is studied which is very dense graphs.

They are defined as graphs with at least $\binom{n}{2} - n^{1 + \beta }$ edges for some $0 \le \beta < 12$.

\section{(G, t)-secret-sharing schemes}

This paper introduces a new family of schemes which is a specialisation of graph secret-sharing schemes.



The forbidden graph is a (G, 2) secret-sharing scheme.

\section{CDS}

This paper uses the concept of Conditional Disclosure of Secrets (CDS) Protocols in the latter part of the paper in the discussion on the graph access structures.

Let $f : X_1 \times \cdots \times X_k \rightarrow {0, 1}$ be a $k$-input function.

A $k$-server CDS protocol $P$ for $f$ with domain of secrets $S$ consists of:

\begin{enumerate}
    \item A finite domain of common random strings $R$, and $k$ finite message domains
    $M_1 , \ldots , M_k$,
    \item Deterministic message computation functions $\text{Enc}_1 , \ldots , \text{Enc}_k$ ,
    where $\text{Enc}_i : X_i \times S \times R \rightarrow M_i$ for every $i \in [k]$ (we also say that $\text{Enc}_i (x_i , s, r)$ is the
    message sent by the $i$-th server to the referee), and
    \item A deterministic reconstruction function $\text{Dec} : X_1 \times \cdots \times X_k \times M_1 \times \cdots \times M_k \rightarrow \{0, 1\}$.
\end{enumerate}

We denote $\text{Enc}(x, s, r) = (\text{Enc}_1 (x_1 , s, r), \ldots, \text{Enc}_k(x_k , s, r))$.

We say that a CDS protocol $P$ is a CDS protocol for a function $f$ if the following two require-
ments hold:

\textbf{Correctness}. For any input $(x_1 , \cdots , x_k ) \in X_1 \times \cdots \times X_k$ for which
$f(x_1, \cdots , x_k ) = 1$, every secret $s \in S$, and every common random string $r \in R$,
$\text{Dec}(x_1 , \cdots , x_k , \text{Enc}_1 (x_1, s, r), \cdots , \text{Enc}_k (x_k , s, r)) = s$.

\textbf{Privacy}. For any input $x = (x_1 , \cdots , x_k ) \in X_1 \times \cdots \times X_k$ for which
$f (x_1, \cdots , x_k ) = 0$ and for every pair of secrets $s$, $s'$ , the distributions $\text{Enc}(x, s, r)$
and $\text{Enc}(x, s' , r)$ are identical, where the distributions are over the choice of $r$
from $R$ at random with uniform distribution.

The message size of a CDS protocol $P$ is defined as the size of largest message
sent by the servers, i.e., $\max_{1 \le i \le k} \{\log |M_i|\}$.

The fact that $(G,t)$-secret-sharing schemes are equivalent to some particular configurations of CDS protocols is used to improve the bounds for the share sizes for the dense graphs.

\section{Main results}

The paper rigorously defines the bounds for the share sizes for the number of secret-sharing schemes.

For the secret-sharing schemes of \textbf{almost all access structures}, the paper proves the following bounds:

\begin{itemize}
    \item The maximum share size for one-bit secret is bounded 
    \begin{itemize}
        \item from the top by $2^{O(\sqrt{n}\log n)}$
        \item from the bottom by $\Omega(\log n)$
    \end{itemize}
    \item The maximum share size for linear schemes is bounded
    \begin{itemize}
        \item from the top by $2^{n/2+o(n)}$
        \item from the bottom by $\Omega(2^{n/3-o(n)}\log q)$
    \end{itemize}
    \item The normalized maximum share size for multi-linear schemes for secrets of size $2^{n^2}$ is bounded
    \begin{itemize}
        \item from the top by $2^{O(\sqrt{n \log n})}$
    \end{itemize}
\end{itemize}

All these results are specified as mathematically proven theorems.

While proving these bounds, two hypotheses from \cite{applebaumAmortization} are proven:

\begin{itemize}
    \item \textbf{SS is short}. Every access structure over n parties is realizable with small information ratio (say $2^{o(n)}$ ).
    \item \textbf{SS is amortizable}. For every access structure over $n$ parties, and every sufficiently long secret $s$, 
            there exists a secret-sharing scheme with small information ratio (e.g., sub-exponential in $n$).
\end{itemize}

For the secret-sharing schemes of \textbf{almost all graphs}, the paper proves the following bounds:

\begin{itemize}
    \item The maximum share size for one-bit secret is bounded 
    \begin{itemize}
        \item from the top by $n^{o(1)}$
        \item from the bottom by $\Omega(\log n)$
    \end{itemize}
    \item The maximum share size for linear schemes is bounded
    \begin{itemize}
        \item from the top by $\tilde{O}(\sqrt{n} \log q)$
    \end{itemize}
    \item The normalized maximum share size for multi-linear schemes for secrets of size $2^{n^2}$ is bounded
    \begin{itemize}
        \item from the top by $\tilde{O}(\log n)$
        \item from the bottom by $\Omega(\log^{1/2} n)$
    \end{itemize}
\end{itemize}

All these results are specified as mathematically proven theorems.

A new family of schemes called $(G, t)$-secret-sharing schemes is introduced.

\section{Discussion of the results}

Authors perform super rigorous analysis on an array of types of secret-sharing schemes.
The paper is super tight on mathematics and is obviously targeted on the specialists in cryptography.

However, this paper can also serve as an overview on the major secret-sharing schemes and their properties.

It seriously looks like multiple papers compressed into one, especially with the result for the dense graphs which looks like a separate work on its own.
As the main value of this paper is summarizing the existing works on share sizes and proving some of the results,
but the result for the dense graphs has been gotten by the novel usage of a CDS scheme.

I admit I didn't understand the importance of the term "almost all access structures" used extensively in the paper.

In general, knowing the theoretical bounds for the share sizes is crucial for the practical implementation of the secret-sharing schemes.
As it's important to know your absolute limits to understand when you need to stop optimizing your code.


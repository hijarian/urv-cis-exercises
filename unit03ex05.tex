\chapter{Unit 3}

\section{Exercise 5}

\subsection{Task}
Prove that the binary version of the Vigenère cipher with $\mathcal{K} = \{0, 1\}^l$ and $\mathcal{M} = \{0, 1\}^{l+1}$ is not perfectly secure.

\subsection{Solution}

First, let's note that the condition of the task means that our key is 1 bit shorter than the plaintext.

Now, a cipher is perfectly secure if:

\begin{equation}
\forall m \in \mathcal{M}, \forall c \in \mathcal{C}: \; P(M = m | C = c) = P(M = m)
\end{equation}

where $\mathcal{C}$ is the set of all possible ciphertexts.

In other words, observing the ciphertext does not give any information about the plaintext.

Let's try to find a counterexample:

\begin{equation}
\exists m \in \mathcal{M}, \exists c \in \mathcal{C}: \; P(M = m | C = c) \neq P(M = m)
\end{equation}

Due to how the Vigenère cipher works, in our case, where the length of the key is smaller than the length of the plaintext, at $l+1$ bit, the ciphertext will be creating by XORing with the first bit of the key.

Let's consider the following ciphertext:

\begin{equation}
c = (b, *, \ldots, *, b) \text{ where } b \in \{0,1\} \text{ and } * \text{ represents any bit}
\end{equation}

(first and last bits are the same).

This means that it's impossible for the first and last bits of the plaintext to be different, because they are created by the same bit of the key.

That is:

\begin{equation}
P(M = (u, *, \ldots, *, v) | C = c) = 0, u \neq v
\end{equation}

but such a plaintext exists:

\begin{equation}
P(M = (u, *, \ldots, *, v)) > 0, u \neq v
\end{equation}

Therefore, the cipher is not perfectly secure. \qed

